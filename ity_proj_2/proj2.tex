\documentclass[a4paper, 11pt, twocolumn]{article}
    \usepackage[left=1.5cm, top=2.5cm, text={18cm, 25cm}]{geometry}
    \usepackage[czech]{babel}
    \usepackage[utf8]{inputenc}
    \usepackage{times}
    \usepackage[IL2]{fontenc}
    \usepackage{amsthm, amssymb, amsmath}
    
    \theoremstyle{definition}
    \newtheorem{definition}{Definice}
    \newtheorem{sentence}{Věta}
    
\begin{document}

\title{Test}
\author{Jan Demel}

\begin{titlepage}
    \begin{center}
        {\Huge\textsc{
            Fakulta informačních technologií\\
            Vysoké učení technické v~Brně\\
        }}
        \vspace{\stretch{0.382}}
		{\LARGE
			Typografie a publikování – 2. projekt \\
			Sazba dokumentů a matematických výrazů\\
		}
		\vspace{\stretch{0.618}}
		{\Large
			2020
			\hfill
			Jan Demel (xdemel01)
		}
    \end{center}
\end{titlepage}

\section*{Úvod}
V této úloze si vyzkoušíme sazbu titulní strany, matematic\-kých vzorců, prostředí a dalších textových struktur obvyklých pro technicky zaměřené texty (například rovnice (\ref{second_equation}) nebo Definice \ref{second_definition} na straně \pageref{second_definition}). Pro vytvoření těchto odkazů používáme příkazy \verb|\label|, \verb|\ref| a \verb|\pageref|.

Na titulní straně je využito sázení nadpisu podle optického středu s využitím zlatého řezu. Tento postup byl
probírán na přednášce. Dále je použito odřádkování se
zadanou relativní velikostí 0.4em a 0.3em.

\section{Matematický text}
Nejprve se podíváme na sázení matematických symbolů
a~výrazů v~plynulém textu včetně sazby definic a~vět s~využitím balíku \verb|amsthm|. Rovněž použijeme poznámku pod čarou s~použitím příkazu \verb|\footnote|. Někdy je vhodné použít konstrukci \verb|${}$| nebo \verb|\mbox{}| která říká, že\linebreak(matematický) text nemá být zalomen. V~následující definici je nastavena mezera mezi jednotlivými položkami
\verb|\item| na 0.05em.

\begin{definition}
    \label{first_definition}
    Turingův stroj \emph{(TS) je definován jako šestice tvaru } $ M = (Q, \Sigma, \Gamma, \delta, q_0, q_F) $, \emph{ kde:}
    \begin{itemize}
        \setlength\itemsep{0.05em}
        \item Q \emph{je konečná množina} vnitřních (řídících) stavů,
        \item $\Sigma$ \emph{je konečná množina symbolů nazývaná} vstupní abeceda, $\Delta \notin \Sigma$,
        \item $\Gamma$ \emph{je konečná množina symbolů,} $\Sigma \subset \Gamma, \Delta \in \Gamma$, \emph{nazývaná} pásková abeceda,
        \item $\delta : (Q\backslash \{q_F\} ) \times \Gamma \rightarrow Q \times (\Gamma \cup \{L, R\})$, \emph{kde} $L, R \notin \Gamma$, \emph{je parciální} přechodová funkce, \emph{a}
        \item $q_0 \in Q$ \emph{je} počáteční stav \emph{a} $q_f \in Q$ \emph{je} koncový stav.
    \end{itemize}
    
    Symbol $\Delta$ značí tzv. \emph{blank} (prázdný symbol), který se vyskytuje na místech pásky, která nebyla ještě použita.
    
    \emph{Konfigurace pásky} se skládá z nekonečného ře\-tězce, který reprezentuje obsah pásky a pozice hlavy~na~tomto řetězci. Jedná se o prvek množiny $\{ \gamma \Delta^\omega$ $|$ $\gamma \in \Gamma^* \} \times \mathbb{N}$\footnote{Pro libovolnou abecedu $\Sigma$ je $\Sigma^\omega$ množina všech \emph{nekonečných} \linebreak řetězců nad $\Sigma$, tj. nekonečných posloupností symbolů ze $\Sigma$.}.
    \emph{Konfiguraci pásky} obvykle zapisujeme jako $\Delta x y z z x \Delta\ldots$ (podtržení značí pozici hlavy). \emph{Konfigurace stroje} je pak dána stavem řízení a konfigurací pásky. Formálně se jedná o prvek množiny $Q \times \{\gamma \Delta \omega | \gamma \in \Gamma^*\}\times \mathbb{N}$.
    
\end{definition}

\subsection{Podsekce obsahující větu a odkaz}
\begin{definition}
    \label{second_definition}
    Řetězec $w$ nad abecedou $\Sigma$ je přijat TS \emph{M jestliže M při aktivaci z počáteční konfigurace pásky \linebreak $\underline{\Delta}w\Delta\ldots$ a počátečního stavu $q_0$ zastaví přechodem do koncového stavu $q_F$, tj. $(q_0, \Delta w \Delta^w, 0) \underset{M}{\overset{*}{\vdash}} (q_F, \gamma, n)$ pro nějaké $\gamma \in \Sigma^*$ a $n \in \mathbb{N}$.}
    
    \emph{Množinu $L(M) = \{w $ $|$ $ w $ je přijat TS M$\} \subseteq \Sigma^*$ nazýváme} jazyk přijímaný TS \emph{M}.
\end{definition}

Nyní si vyzkoušíme sazbu vět a důkazů opět s použitím
balíku \verb|amsthm|.

\begin{sentence}
	\emph{Třída jazyků, které jsou přijímány TS, odpovídá} rekurzivně vyčíslitelným jazykům.
\end{sentence}

\begin{proof}
	V důkaze vyjdeme z Definice \ref{first_definition} a \ref{second_definition}.
\end{proof}

\section{Rovnice}
Složitější matematické formulace sázíme mimo plynulý
text. Lze umístit několik výrazů na jeden řádek, ale pak je
třeba tyto vhodně oddělit, například příkazem \verb|\quad|.

$$\sqrt[i]{x^3_i} \quad \text{kde } {x_i} \text{ je } i\text{-té sudé číslo} \quad {y^{2\cdot y_i}_i}\neq {y^{y^{y_i}_i}_i}$$

V rovnici (\ref{first_equation}) jsou využity tři typy závorek s různou explicitně definovanou velikostí.

\begin{eqnarray}
	\label{first_equation} x & = & \bigg\{ \Big( \big[ a + b \big] * c \Big)^d \oplus 1 \bigg\} \\
	\label{second_equation} y & = & \lim_{x \to \infty} \frac{\sin^2x + \cos^2x}{\frac{1}{\log_{10}x}} 
\end{eqnarray}

V této větě vidíme, jak vypadá implicitní vysázení limity $\lim_{n \rightarrow \infty} f(n)$ v normálním odstavci textu. Podobně je to i s dalšími symboly jako $\sum_{i=1}^{n} 2^{i}$ či $\cap_{A\in \mathcal{B}}A$. V případě vzorců ${\lim\limits_{x \rightarrow \infty} f(n)}\ \text{a}\ \sum\limits_{i=1}^{n} 2^{i}$ jsme si vynutili méně úspornou sazbu příkazem \verb|\limits|.

\section{Matice}
Pro sázení matic se velmi často používá prostředí \verb|array| a závorky (\verb|\left|, \verb|\right|).

$$\left(\begin{array}{ccc}a + b & \widehat{\xi + \omega} &\hat{\pi}\\\vec{\textbf{a}} & \overleftrightarrow{AC} & \beta \\\end{array}\right)= 1 \Longleftrightarrow \mathbb{Q} = \mathcal{R}$$

Prostředí \verb|array| lze úspěšně využít i jinde.

$$\binom{n}{k} =\left\{\begin{array}{ll}\hspace{0.5cm}0 & \text{pro } k \ < 0 \text{ nebo } k > n \\\frac{n!}{k! (n - k)!} & \text{pro } 0 \leq k \leq n.\end{array}\right.$$

\end{document}